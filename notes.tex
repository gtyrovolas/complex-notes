%!TEX output_directory = .aux
%!TEX copy_output_on_build(true)

\documentclass[11pt,a4paper, titlepage]{article}
\usepackage[a4paper, total={6.5in, 8in}]{geometry}
\usepackage[utf8]{inputenc}
\usepackage{amsfonts}
\usepackage{amssymb}
\usepackage{amsmath}
\usepackage{mathtools}
\usepackage{amsthm}

\title{Metric Spaces and Complex Analysis}
\author{Giannis Tyrovolas}
\date{September 18, 2020}
\newtheorem{theorem}{Theorem}[section]
\newtheorem{prop}[theorem]{Proposition}
\newtheorem*{remark}{Remark}
\DeclarePairedDelimiter\abs{\lvert}{\rvert}
\DeclarePairedDelimiter\norm{\lVert}{\rVert}

\theoremstyle{definition}
\newtheorem{definition}[theorem]{Definition}
\newtheorem{example}[theorem]{Example}
\newtheorem{corollary}[theorem]{Corollary}
\newtheorem{lemma}[theorem]{Lemma}
\newtheorem{proposition}[theorem]{Proposition}
\newtheorem*{idea}{Idea}

\begin{document}

\maketitle

\section{Intro}

\begin{definition}[Domain]
A domain usually denoted $U$ is an open, connected subset of the complex numbers.
\end{definition}

\begin{theorem}[Cauchy's Theorem]

Let $ f \colon U \longrightarrow \mathbb{C}$ holomorphic on a domain $U$. Then for all closed paths $\gamma$:
\[
	\int_\gamma f(z) dz = 0
\]

\end{theorem}

\begin{theorem}[Deformation Theorem]

Let $f \colon U \longrightarrow \mathbb{C}$ be holomorphic on domain $U$. Let two closed paths $\gamma_1, \gamma_2 $ be homotopic. Then:

\[
	\int_{\gamma_1} f = \int_{\gamma_2} f
\]	

\end{theorem}

\begin{theorem}[Cauchy's Integral Formula]

Let $f \colon U \longrightarrow \mathbb{C}$ holomorphic on and inside a simple, closed, positively oriented curve $\gamma$. Then for all points $a$ on the interior of $\gamma$:

\[
	f(a) = \frac{1}{2\pi i} \int_\gamma \frac{f(w)}{w-a} \mathrm{d}w
\]	

\end{theorem}

\begin{proof}

Since the interior of $\gamma$ is open there is an $r > 0$ such that $D(a,r)$ is contained in the interior of $\gamma$. Then by the deformation theorem:

\[
	\int_\gamma \frac{f(w)}{w-a} \mathrm{d}w = \int_{\gamma(a,r)} \frac{f(w)}{w-a} \mathrm{d}w = I
\]

Without loss of generality let $g(w) = f(w - a)$. Then, for $u = w - a$


\begin{align*}
	I &= \int_{\gamma(a,r)} \frac{f(w)}{w-a} \mathrm{d}w \\
	  &= \int_{\gamma(0,r)} \frac{g(u)}{u} \mathrm{d}u \\
	  &= \int_0^{2\pi} \frac{g(re^{i \theta})}{re^{i \theta}} i r e^{i \theta} d \theta \\
	  &= i \int_0^{2\pi} g(re^{i \theta}) d \theta 
\end{align*}

Hence, 
\begin{align*}
\abs{I - 2 \pi i \, g(0)} &= \abs{i \int_0^{2\pi} g(re^{i \theta}) - g(0) d \theta} \\
					      &\leqslant 2 \pi  \sup_{\theta \in [0,2 \pi)} \abs{g(re^{i \theta}) - g(0)}  \\
					      & \rightarrow 0 
\end{align*}
as $r$ tends to $0$ by the continuity of $f$.

Hence $f(a) = g(0) = \frac{1}{2 \pi i} I$ and $I = \int_{\gamma(a,r)} \frac{f(w)}{w-a} \mathrm{d}w$

\end{proof}

\begin{theorem}[Taylor's Theorem]

All holomorphic functions on a domain can be expressed as a power series. For $f \colon U \longrightarrow \mathbb{C}$ holomorphic on domain $U$ and for $a \in U$, $D(a,r) \subseteq U$

\[
	f(z) = 	\sum_{n=0}^\infty c_n (z-a)^n
\]
where:

\[
	c_n = \frac{1}{2 \pi i} \int_{\gamma(a,r)} \frac{f(w)}{(w-a)^{n+1}} = \frac{f^{(n)}(a)}{n!}
\]
\end{theorem}


\begin{theorem}[Liouville's Theorem]
Let $f$ holomorphic on $\mathbb{C}$ and $f$ bounded. Then $f$ is constant.
\end{theorem}

\begin{corollary}
For $f$ entire, $f(\mathbb{C})$ is dense in $\mathbb{C}$ (i.e. $\bar{f(\mathbb{C})} = \mathbb{C})$
\end{corollary}

\begin{theorem}[Picard's Little Theorem]
For $f$ non-constant entire, $f(\mathbb{C}) = \mathbb{C} \textrm{ or } \mathbb{C} \setminus \{z\} $
\end{theorem}

\begin{theorem}[Fundamental Theorem of Algebra]

Let $p$ be a non-constant polynomial with complex coefficients. Then there exists $a \in \mathbb{C}$ such that $p(a) = 0$.

\end{theorem}

\begin{theorem}[Morera's Theorem]
Let $f$ continuous on a domain $U$ and for all closed paths $\gamma$ in $U$
\[
	\int_\gamma f(z) dz = 0
\]
Then $f$ is holomorphic.


\end{theorem}

\begin{proof}


\end{proof}

\begin{theorem}[Identity Theorem]
Let $f$ holomorphic on domain $U$ let $S = f^{-1}({0})$. If S contains one of it's limit points then $f$ is identically zero.
\end{theorem}

\begin{theorem}[Counting Zeroes]

Let $f$ holomorphic inside and on a positively oriented closed path $\gamma$. Then the sum of zeroes counting their multiplicity is:

\[
	\frac{1}{2 \pi i} \int_\gamma \frac{f'(w)}{f(w)}dw 
\]


\end{theorem}

\begin{theorem}[Laurent's Theorem]

Let $f$ be a function holomorphic on ${ z \in  \mathbb{C} | R < \abs{z - a} < S}$. Then, 

\[
 	f(z) = \sum_{n = -\infty}^{\infty} c_n (z-a)^n
 \] 
For:

\[
	c_n = \frac{1}{2 \pi i} \int_{\gamma(a,r)} \frac{f(w)}{(w-a)^{n+1}}dw
\]
\end{theorem}



\end{document}

