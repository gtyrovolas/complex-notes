%!TEX output_directory = .aux
%!TEX copy_output_on_build(true)

\documentclass[11pt,a4paper, titlepage]{article}
\usepackage[a4paper, total={6.5in, 8in}]{geometry}
\usepackage[utf8]{inputenc}
\usepackage{amsfonts}
\usepackage{amssymb}
\usepackage{amsmath}
\usepackage{mathtools}
\usepackage{amsthm}

\title{Metric Spaces and Complex Analysis}
\author{Giannis Tyrovolas}
\date{September 18, 2020}
\newtheorem{theorem}{Theorem}[section]
\newtheorem{prop}[theorem]{Proposition}
\newtheorem*{remark}{Remark}
\DeclarePairedDelimiter\abs{\lvert}{\rvert}
\DeclarePairedDelimiter\norm{\lVert}{\rVert}

\theoremstyle{definition}
\newtheorem{definition}[theorem]{Definition}
\newtheorem{example}[theorem]{Example}
\newtheorem{corollary}[theorem]{Corollary}
\newtheorem{lemma}[theorem]{Lemma}
\newtheorem{proposition}[theorem]{Proposition}
\newtheorem*{idea}{Idea}

\begin{document}

\maketitle

\section{Metric Spaces}
\begin{definition}[Metric Space]
A metric space $M = (X, d)$ is a set equipped with a function $d \colon X \times X \longrightarrow \mathbb{R}$ such that:

\begin{enumerate}
	\item $d(x,y) \geqslant 0 $ and $d(x,y) = 0 \iff x = y$
	\item $d(x,y) = d(y,x)$
	\item $d(x,z) \leqslant d(x,y) + d(y,z)$
\end{enumerate}
\end{definition}

\begin{definition}[Continuity]
A function $f \colon X \longrightarrow Y$ is continuous at $x_0 \in X$ when $\forall \varepsilon > 0 \; \exists \delta > 0$ such that $\forall x \in B(x_0, \delta)$, $f(x) \in B(f(x_0), \varepsilon)$
\end{definition} 

\begin{definition}[Uniform Continuity]
A function $f \colon X \longrightarrow Y$ is uniformly continuous if $\forall \varepsilon >0 \; \exists \delta > 0$ such that $\forall z \in B(x,\delta)$ $f(z) \in B(f(x), \varepsilon)$
\end{definition}

\begin{definition}[Convergence]
A series $(x_n)$ converges in a metric space $X$ if there is an $x_0 \in X$ such that for all $\varepsilon > 0$ there is an $N \in \mathbb{N}$ such that for all $ n > N$ $d(x_n, x_0) < \varepsilon$
\end{definition}

\begin{lemma}[Sequential Continuity]

A function $f \colon X \longrightarrow Y$ is continuous at $a \in X$ if and only if for every sequence $(x_n) \to a$, $(f(x_n)) \to f(a)$

\end{lemma}

\begin{definition}[Norm]
Let $V$ a vector space. Then $\norm{.} \colon V \longrightarrow \mathbb{R}$ is a norm if:
\begin{enumerate}
	\item $\norm{v} \geqslant 0$ and $\norm{v} = 0 \iff v = 0_V $
	\item $\norm{ \lambda v } = \abs{\lambda} \norm{v}$
	\item $\norm{x + y} \leqslant \norm{x} + \norm{y}$
\end{enumerate}
\end{definition}

\begin{definition}[Open Set]
A set $U \subseteq X$ is open if $\forall x \in U$, there is an $\varepsilon > 0$ such that $B(x, \varepsilon) \subseteq U$. 
\end{definition}

\begin{theorem}[Topological Continuity]
A function $f \colon X \longrightarrow Y$ is continuous if and only if the pre-image of every open set is open.
\end{theorem}

\begin{definition}[Interior]
The interior of $S$ is the largest open subset of $S$, defined as:
\[
	int(S) = \bigcup_{ U \subseteq S \textrm{,  } U \textrm{ open}} U
\]
\end{definition}

\begin{definition}[Closure]
The closure of a set $S$ is the smallest closed subset containing $S$:

\[
	\overline{S} = \bigcap_{S \subseteq C, \,  C\textrm{ closed}} C
\]
\end{definition}

\begin{lemma}
A function is continuous if and only if $f(\overline{S}) \subseteq \overline{f(S)}$
\end{lemma}

\section{Complex Exponential}

The following power series define the complex exponential and trigonometric functions:

\begin{align*}
	\exp z &= \sum_{n = 0}^\infty \frac{z^n}{n!}, & \sin z &= \sum_{k = 0}^\infty (-1)^k\frac{z^{(2k+1)}}{(2k + 1)!},& \cos z &= \sum_{k = 0}^\infty (-1)^k\frac{z^{2k}}{(2k)!} \\
 	&& \sinh &= \sum_{k = 0}^\infty \frac{z^{(2k+1)}}{(2k + 1)!}, &\cosh &= \sum_{k = 0}^\infty \frac{z^{2k}}{(2k)!}
\end{align*}

Note:
\begin{align*}
\sin z &= \frac{e^{iz} - e^{-iz}}{2i}, & \cos z &= \frac{e^{iz} + e^{-iz}}{2} \\
\sinh z &= \frac{e^{z} - e^{-z}}{2}, & \cosh z &= \frac{e^{z} + e^{-z}}{2}
\end{align*}

And
\[
	\exp i \theta = \cos \theta + i \sin \theta
\]
	


\section{Holomorphic Functions}

\begin{definition}[Domain]
A domain usually denoted $U$ is an open, connected subset of the complex numbers.
\end{definition}

\begin{theorem}[Cauchy's Theorem]

Let $ f \colon U \longrightarrow \mathbb{C}$ holomorphic on a domain $U$. Then for all closed paths $\gamma$:
\[
	\int_\gamma f(z) dz = 0
\]

\end{theorem}

\begin{theorem}[Deformation Theorem]

Let $f \colon U \longrightarrow \mathbb{C}$ be holomorphic on domain $U$. Let two closed paths $\gamma_1, \gamma_2 $ be homotopic. Then:

\[
	\int_{\gamma_1} f = \int_{\gamma_2} f
\]	

\end{theorem}

\begin{theorem}[Cauchy's Integral Formula]

Let $f \colon U \longrightarrow \mathbb{C}$ holomorphic on and inside a simple, closed, positively oriented curve $\gamma$. Then for all points $a$ on the interior of $\gamma$:

\[
	f(a) = \frac{1}{2\pi i} \int_\gamma \frac{f(w)}{w-a} \mathrm{d}w
\]	

\end{theorem}

\begin{theorem}[Taylor's Theorem]

All holomorphic functions on a domain can be expressed as a power series. For $f \colon U \longrightarrow \mathbb{C}$ holomorphic on domain $U$ and for $a \in U$, $D(a,r) \subseteq U$

\[
	f(z) = 	\sum_{n=0}^\infty c_n (z-a)^n
\]
where:

\[
	c_n = \frac{1}{2 \pi i} \int_{\gamma(a,r)} \frac{f(w)}{(w-a)^{n+1}} = \frac{f^{(n)}(a)}{n!}
\]
\end{theorem}


\begin{theorem}[Liouville's Theorem]
Let $f$ holomorphic on $\mathbb{C}$ and $f$ bounded. Then $f$ is constant.
\end{theorem}

\begin{corollary}
For $f$ entire, $f(\mathbb{C})$ is dense in $\mathbb{C}$ (i.e. $\overline {f(\mathbb{C})} = \mathbb{C})$
\end{corollary}

\begin{theorem}[Picard's Little Theorem]
For $f$ non-constant entire, $f(\mathbb{C}) = \mathbb{C} \textrm{ or } \mathbb{C} \setminus \{z\} $
\end{theorem}

\begin{theorem}[Fundamental Theorem of Algebra]

Let $p$ be a non-constant polynomial with complex coefficients. Then there exists $a \in \mathbb{C}$ such that $p(a) = 0$.

\end{theorem}

\begin{theorem}[Morera's Theorem]
Let $f$ continuous on a domain $U$ and for all closed paths $\gamma$ in $U$
\[
	\int_\gamma f(z) dz = 0
\]
Then $f$ is holomorphic.
\end{theorem}


\begin{theorem}[Identity Theorem]
Let $f$ holomorphic on domain $U$ let $S = f^{-1}({0})$. If S contains one of it's limit points then $f$ is identically zero.
\end{theorem}

\begin{theorem}[Counting Zeroes]
Let $f$ holomorphic inside and on a positively oriented closed path $\gamma$. Then the sum of zeroes counting their multiplicity is:
\[
	\frac{1}{2 \pi i} \int_\gamma \frac{f'(w)}{f(w)}dw 
\]
\end{theorem}

\begin{theorem}[Laurent's Theorem]

Let $f$ be a function holomorphic on ${ z \in  \mathbb{C} | R < \abs{z - a} < S}$. Then, 
\[
 	f(z) = \sum_{n = -\infty}^{\infty} c_n (z-a)^n
 \] 
For:

\[
	c_n = \frac{1}{2 \pi i} \int_{\gamma(a,r)} \frac{f(w)}{(w-a)^{n+1}}dw
\]
\end{theorem}



\end{document}

